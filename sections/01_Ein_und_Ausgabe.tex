\section{Ein- und Ausgabe}
Um die C++ Ein- und Ausgabe nutzen zu k�nnen, muss man die Bibliothek iostream einbinden. Das geschieht mit:
\lstinputlisting[language=C++,tabsize=2]{code/iostream.cpp}
Danach m�ssen die Befehle daraus bekannt gegeben werden, da sie sich in einem speziellen Namensraum befinden. Was Namensr�ume sind und wof�r man sie einsetzt, wird sp�ter erkl�rt. Um nun die Ein- und Ausgabebefehle nutzen zu k�nnen, muss man dem Compiler sagen: Benutze den Namensraum std. Dies erreicht man mit folgender Zeile:
\lstinputlisting[language=C++,tabsize=2]{code/using_namespace_std.cpp}
	\subsection{Streamkonzept}
		\begin{compactitem}
			\item Ein Stream repr�sentiert einen sequentiellen Datenstrom.
			\item Die Operatoren auf dem Stream sind << und >>.\newline
			F�r vordefinierte Datentypen sind diese Operatoren schon definiert, f�r eigene selbstdefinierte Klassen
			k�nnen diese Operatoren �berladen werden (folgt sp�ter).
			\item C++ stellt 4 Standardstr�me zur Verf�gung:
				\begin{compactitem}
					\item cin: Standard-Eingabestrom, normalerweise die Tastatur
					\item cout: Standard-Ausgabestrom, normalerweise der Bildschirm
					\item cerr: Standard-Fehlerausgabestrom, normalerweise der Bildschirm
					\item clog: mit cerr gekoppelt
				\end{compactitem}
			\item Alle diese Str�me k�nnen auch mit einer Datei verbunden werden.	
		\end{compactitem}
	\subsection{Ausgabe}
	Die Klasse ostream stellt Methoden zur Ausgabe aller vordefinierten Datentypen (char, bool, int, etc) zur Verf�gung. Alle Ausgabemethoden sind �berladene Versionen des Operators <<. Die verschiedenen Versionen unterscheiden sich dabei in ihren Parametern und haben etwa folgende Schnittstelle:
	\lstinputlisting[language=C++,tabsize=2]{code/ausgabe_1.cpp}
	Nutzung mit cout (vordefiniertes Objekt der Klasse ostream):
	\lstinputlisting[language=C++,tabsize=2]{code/ausgabe_2.cpp}
	\subsection{Eingabe}
	Die Eingabe ist �hnlich organisiert wie die Ausgabe. Die Klasse istream ist die Abstraktion eines Eingabestroms und stellt unter anderem folgende M�glichkeiten zur Verf�gung: 
	\lstinputlisting[language=C++,tabsize=2]{code/eingabe_1.cpp}
	Nutzung mit cin (vordefiniertes Objekt der Klasse istream):
	\lstinputlisting[language=C++,tabsize=2]{code/eingabe_2.cpp}
	\subsection{Formatierte Ein- und Ausgabe}
	ios, eine Basisklasse von iostream, stellt verschiedene M�glichkeiten (Format Flags) vor, um die Ein- und Ausgabe
	zu beeinflussen:
	\newline \newline
	\textbf {boolalpha:} bool-Werte werden textuell ausgegeben
	\subsection{Diverse Beispiele}
	\lstinputlisting[language=C++,tabsize=2]{code/ausgabe_3.cpp}
	