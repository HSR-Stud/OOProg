\section{Ein- und Ausgabe}
Um die C++ Ein- und Ausgaben nutzen zu k�nnen, muss man die Bibliothek iostream einbinden. Das geschieht mit:
\lstinputlisting[language=C++,tabsize=2]{code/iostream.cpp}
Danach m�ssen die Befehle daraus bekannt gegeben werden, da sie sich in einem speziellen Namespace befinden. Um nun die Ein- und Ausgabebefehle nutzen zu k�nnen, muss man dem Compiler sagen: Benutze den Namenspace std. Man k�nnte vor jeden Ein- Ausgabebefehl std:: schreiben. Da dies aber m�hsam ist, kann man mit folgender Zeile dieses Problem umgehen:
\lstinputlisting[language=C++,tabsize=2]{code/using_namespace_std.cpp}
	
	\subsection{Streamkonzept}
		\begin{compactitem}
			\item Ein Stream repr�sentiert einen sequentiellen Datenstrom.
			\item Die Operatoren auf dem Stream sind << und >>.\newline
			F�r vordefinierte Datentypen sind diese Operatoren schon definiert, f�r eigene selbstdefinierte Klassen
			k�nnen diese Operatoren �berladen werden (folgt sp�ter).
			\item C++ stellt 4 Standardstr�me zur Verf�gung:
				\begin{compactitem}
					\item cin: Standard-Eingabestrom, normalerweise die Tastatur
					\item cout: Standard-Ausgabestrom, normalerweise der Bildschirm
					\item cerr: Standard-Fehlerausgabestrom, normalerweise der Bildschirm
					\item clog: mit cerr gekoppelt
				\end{compactitem}
			\item Alle diese Str�me k�nnen auch mit einer Datei verbunden werden.	
		\end{compactitem}
	
	\subsection{Ausgabe}
	Die Klasse ostream stellt Methoden zur Ausgabe aller vordefinierten Datentypen (char, bool, int, etc) zur Verf�gung. Alle Ausgabemethoden sind �berladene Versionen des Operators <<. Die verschiedenen Versionen unterscheiden sich dabei in ihren Parametern und haben etwa folgende Schnittstelle:
	\lstinputlisting[language=C++,tabsize=2]{code/ausgabe_1.cpp}
	Nutzung mit cout (vordefiniertes Objekt der Klasse ostream):
	\lstinputlisting[language=C++,tabsize=2]{code/ausgabe_2.cpp}
	endl bewirkt bei einer Bildschirmausgabe einen Sprung zum Anfang der n�chsten Zeile und das Leeren des Puffers auf das Ausgabemedium. Es bewirkt eigentlich das gleiche wie die Zeile:
	\lstinputlisting[language=C++,tabsize=2]{code/endl.cpp}	 
	flush bewirkt das Leeren des Puffers auf das Ausgabemedium.
	\subsubsection{Ausgabebeispiele}
		\lstinputlisting[language=C++,tabsize=2]{code/ausgabe_3.cpp}
		\vspace{0.5cm}
		\lstinputlisting[language=C++,tabsize=2]{code/ausgabe_4.cpp}\newpage
		
		
	\subsection{Eingabe}
	Die Eingabe ist �hnlich organisiert wie die Ausgabe. Die Klasse istream ist die Abstraktion eines Eingabestroms und stellt unter anderem folgende M�glichkeiten zur Verf�gung: 
	\lstinputlisting[language=C++,tabsize=2]{code/eingabe_1.cpp}
	Nutzung mit cin (vordefiniertes Objekt der Klasse istream):
	\lstinputlisting[language=C++,tabsize=2]{code/eingabe_2.cpp}	
		\subsubsection{Eingabebeispiele}
			\lstinputlisting[language=C++,tabsize=2]{code/eingabe_3.cpp}
		\vspace{0.5cm}
	\subsection{Formatierte Ein- und Ausgabe: Manipulatoren ohne Parameter}
	ios, eine Basisklasse von iostream, stellt verschiedene M�glichkeiten (Format Flags) vor, um die Ein- und Ausgabe zu beeinflussen.\newline
		\begin{minipage}[t]{9 cm}
			\subsubsection{boolalpha} 
				bool-Werte werden textuell ausgegeben.
				\lstinputlisting[language=C++,tabsize=2]{code/boolalpha.cpp}
		\end{minipage}
		\hspace*{0.8cm}	
		\begin{minipage}[t]{9 cm}
			\subsubsection{showbase} 
				Zahlenbasis wird gezeigt.
				\lstinputlisting[language=C++,tabsize=2]{code/showbase.cpp}
		\end{minipage}
		\newline
		\begin{minipage}[t]{9 cm}
			\subsubsection{showpoint} 
				Dezimalpunkt wird immer ausgegeben.
				\lstinputlisting[language=C++,tabsize=2]{code/showpoint.cpp}
		\end{minipage}
		\hspace*{0.8cm}			
		\begin{minipage}[t]{9 cm}
			\subsubsection{showpos} 
				Vorzeichen bei positiven Zahlen wird angezeigt.
				\lstinputlisting[language=C++,tabsize=2]{code/showpos.cpp}
		\end{minipage}
		\newline
		\begin{minipage}[t]{9 cm}
			\subsubsection{skipws} 
				F�hrende Whitespaces werden nicht angezeigt.
		\end{minipage}	
		\hspace*{0.8cm}					
		\begin{minipage}[t]{9 cm}
			\subsubsection{unitbuf} 
				Leert Buffer des Outputstreams nach Schreiben.
		\end{minipage}
		\vspace{0.6cm}		
		\begin{minipage}[t]{9 cm}
			\subsubsection{uppercase} 
				Alle Kleinbuchstaben in Grossbuchstaben wandeln.
				\lstinputlisting[language=C++,tabsize=2]{code/uppercase.cpp}
		\end{minipage}
		\hspace*{0.8cm}
		\begin{minipage}[t]{9 cm}
			\subsubsection{dec, hex, oct} 
				Ausgabe erfolgt in dezimal, hexadezimal oder oktal.
				\lstinputlisting[language=C++,tabsize=2]{code/dec_hex_oct.cpp}
<<<<<<< HEAD
		\end{minipage}			
=======
		\end{minipage}	
		\newline		
>>>>>>> refs/remotes/origin/master
		\begin{minipage}[t]{9 cm}
			\subsubsection{fixed, scientific} 
				Gleitkommazahlen im Fixpunktformat oder wissenschaftlich.
				\lstinputlisting[language=C++,tabsize=2]{code/fixed_scientific.cpp}
		\end{minipage}
		\hspace*{0.8cm}			
		\begin{minipage}[t]{9 cm}
			\subsubsection{internal, left, right} 
				Ausgabe innerhalb Feld bzw. links oder rechtsb�ndig.
				\lstinputlisting[language=C++,tabsize=2]{code/internal_left_right.cpp} 
		\end{minipage}

	\subsection{Formatierte Ein- und Ausgabe: Manipulatoren mit Parameter} 
		Hier muss zwingend <iomanip> eingebunden werden!\newline\newline
			\begin{minipage}[t]{9 cm}
				\subsubsection{setw()} 
					Angabe der Feldbreite.
					\lstinputlisting[language=C++,tabsize=2]{code/setw.cpp} 
			\end{minipage}
			\hspace*{0.8cm}
			\begin{minipage}[t]{9 cm}
				\subsubsection{setprecision()} 
					Angabe der Genauigkeit einer Zahl.
					\lstinputlisting[language=C++,tabsize=2]{code/setprecision.cpp} 
			\end{minipage}
			\newline
			\begin{minipage}[t]{9 cm}
				\subsubsection{setfill()} 
					Kann ein beliebiges F�llzeichen verwendet werden.
					\lstinputlisting[language=C++,tabsize=2]{code/setfill.cpp} 
			\end{minipage}						
	