\section{G�ltigkeitsbereiche, Namensr�ume und Sichtbarkeit \verweiscpp{9}}
	\subsection{Sichtbarkeit \verweiscpp{9.1.1}}
		\lc{C++} kennt verschiedene G�ltigkeitsbereiche: 
		\begin{compactitem}
			\item Blockanweisungen f�hren einen eigenen G�ltigkeitsbereich ein, den so genannten lokalen G�ltigkeitsbereich oder \lc{Local Scope}. Alle dort deklarierten Bezeichner gelten genau in diesem Block, genauer gesagt von ihrer Deklaration an bis zum Ende des aktuellen Blocks. Unter diesen Punkt fallen auch die Kontrollanweisungen \lc{if}, \lc{switch}, \lc{for} sowie \lc{while}.\\ \ \\
			\item Der G�ltigkeitsbereich von Funktionsprototypen erstreckt sich bis ans Ende der Deklaration und umfasst die Funktionsparameter. Der G�ltigkeitsbereich von Funktionen erstreckt sich �ber die gesamte Funktion.
			\item Die so genannten Namensr�ume (\lc{Namespaces}) sind eigene G�ltigkeitsbereiche, die alle darin deklarierten Bezeichner umfassen. Ein Bezeichner, der in einem Namensraum deklariert ist, gilt von seiner Deklaration bis an das Ende des Namensraums.
			\item Jede Klasse hat einen eigenen G�ltigkeitsbereich (\lc{Class Scope}), der sich �ber die gesamte Klasse erstreckt. Ein Klassenelement gilt in seiner Klasse von seiner Deklaration an bis zum Ende der Klassendeklaration und kann nur in Verbindung mit einer entsprechenden Variablen dieses Klassentyps verwendet werden.
		\end{compactitem}
		
	\subsection{Namensr�ume \verweiscpp{9.1.2}}
		\begin{minipage}[t]{4.5 cm}
			\lstinputlisting[language=C++,tabsize=2]{code/namespace.cpp}
		\end{minipage}
		\begin{minipage}[t]{14.5 cm}
			Namensr�ume sind G�ltigkeitsbereiche, in denen beliebige Bezeichner (Variablen, Klassen, Funktionen, andere Namensr�ume, Typen, etc.) deklariert werden k�nnen.
			\begin{compactitem}
				\item Ein Namensraum kann deklariert werden. Alle enthaltenen Objekte werden diesem Namensraum zugeordnet. Auf Bezeichner eines Namensraumes kann mit dem Scope Operator \lc{::} zugegriffen werden.
				\item Einem Namensraum kann ein so genannter Alias zugeordnet werden, �ber den er angesprochen wird. \\
				\lc{namespace FBSSLIB = Financial\_Branch\_and\_System\_Service\_Library;}
				\item Eine so genannte \lc{Using}-Deklaration erlaubt den direkten Zugriff auf einen Bezeichner eines Namensraumes. \\
				\lc{using MyLib1::foo;} \\
				\lc{foo();}
				\item Mit einer so genannten \lc{Using}-Direktive kann auf alle Bezeichner eines Namensraums direkt zugegriffen werden. \\
				\lc{using namespace MyLib1;} \\
				\lc{foo();}			
			\end{compactitem}
		\end{minipage}
		
	\subsection{Deklarationen \verweiscpp{9.2}}
		\subsubsection{Speicherklassenattribute \verweiscpp{9.2.1}}
			\begin{compactitem}
				\item \lc{auto}: gilt als Standard wenn nichts anderes steht. G�ltigkeitsbereich der \lc{auto} Variablen ist innerhalb des Blockes in dem sie deklariert wurde.
				\item \lc{register}: Hinweis an den Compiler m�glichst die Variable in einem Register abzulegen.
				\item \lc{static}: Variablen leben von ihrer Deklaration bis zum Programmende. Geeignet um zum Bsp. Funktionsaufrufe zu z�hlen anstatt mit globaler Variable.
				\item \lc{extern}: Zugriff auf eine \lc{static} Variable in einem anderen File, welches zu einem gesamten Programm gelinkt wurde.
				\item \lc{mutable}: Klassenelemente mit \lc{const} oder \lc{static} Attributen k�nnen nachtr�glich ver�ndert werden.
			\end{compactitem}
	
		\subsubsection{Typqualifikatoren \verweiscpp{9.2.2}}
		\begin{compactitem}
			\item \lc{const}: Objekte d�rfen nicht ver�ndert werden. RValues.
			\item \lc{volatile}: Objekte werden evtl. von Aussen im Programmverlauf ver�ndert, und d�rfen daher vom Compiler nicht zu Optimierungszwecken zwischengespeichert werden. Sie werden immer aus dem Hauptspeicher eingelesen.
		\end{compactitem}
		
		\subsubsection{Funktionsattribute \verweiscpp{9.2.3}}
		\begin{compactitem}
			\item \lc{inline}: Compileranweisung den Funktionsinhalt einer \lc{inline}-Funktion direkt an die Aufrufstelle zu substituieren. Laufzeitoptimierung (kein wirklicher Funktionsaufruf)
			\item \lc{virtual}: Wird im Zusammenhang mit Klassen gebraucht.
			\item \lc{explicit}: Wird im Zusammenhang mit Klassen gebraucht.
		\end{compactitem}
		
		\subsubsection{typedef \verweiscpp{9.2.4}}
			\begin{minipage}[t]{10 cm}
				\vspace*{-0.5cm}\lstinputlisting[language=C,tabsize=2]{code/typedef.c}
			\end{minipage}
			\begin{minipage}[t]{9 cm}
				Das Schl�sselwort \lc{typedef} erm�glicht die Einf�hrung neuer Bezeichner, die dann im Programm anstelle von anderen Typen verwendet werden k�nnen. \lc{typedef} f�hrt allerdings keine neuen Typen, sondern Synonyme f�r einen existierenden Datentyp ein.
			\end{minipage}
			
	\subsection{Initialisierung von Objekten \verweiscpp{9.3}}
		 \vspace*{-0.29cm}\lstinputlisting[language=C,tabsize=2]{code/initial.c}
		 Folgende Regeln m�ssen beachtet werden:
		 \begin{compactitem}
		 	\item Alle Initialisatoren m�ssen Konstantenausdr�cke sein.
		 	\item Konstantenausdr�cke m�ssen eine Konstante oder die Adresse eines externen oder statischen Objektes plus minus einer Konstante liefern.
		 	\item Bei Konstantenausdr�cken d�rfen nur un�re und bin�re Operatoren sowie Funktionen verwendet werden.
		 	\item Die Werte von Vektoren und strukturierten Datentypen werden durch die Angabe der einzelnen Komponenten in geschwungenen Klammern festgelegt.
		 \end{compactitem}
		 
	\subsection{Type-cast \verweiscpp{9.4}}
		\subsubsection{Standard-Typumwandlung \verweiscpp{9.4.1}}
			Ausdr�cke werden bei einer Zuweisung automatisch in den erwarteten Typ umgewandelt.
			\lstinputlisting[language=C,tabsize=2]{code/typecast.c} 
			
	\subsubsection{explizite-Typumwadlung \verweiscpp{9.4.2}}
	 	\textbf{M�gliche Umwandlungen im C-Stil: (Problematisch)}
		\lstinputlisting[language=C,tabsize=2]{code/typecast2.c}
	 	\textbf{neue Typumwandlungen in C++:}
		\lstinputlisting[language=C,tabsize=2]{code/constcast.c}