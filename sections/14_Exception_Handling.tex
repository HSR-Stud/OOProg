\section{Exception Handling \verweiscpp{15}}
	\begin{minipage}[t]{7 cm}
		\subsection{Exception vs. Error}
			\begin{compactitem}
				\item Error: Abweichung zur Spezifikation ("falsch implementiert").	Errors sollten bei der Verifikation (Testen) entdeckt und eliminiert werden.
				\item Exception: abnormale (aber vorhersehbare und m�gliche) Bedingung bei der
				Programmausf�hrung.
			\end{compactitem}			
	\end{minipage}
	\hspace*{0.5cm}
	\begin{minipage}[t]{11 cm}
			\subsubsection{M�gliche Reaktionen auf Ausnahmen \verweiscpp{15.1}}
				\begin{compactitem}
				 	\item Ignorieren: Motto: Augen zu und durch, eine sehr risikoreiche Variante.
				 	\item Programmabbruch: 	Merkt immerhin, dass etwas nicht in Ordnung ist, die Reaktion ist aber unbefriedigend. Ist Exception Detection aber nicht eigentlich Exception Handling.
				 	\item Exceptioncodes (nicht Fehlercodes): Funktionen geben als R�ckgabewert, als Parameter oder global einen	Ausnahmecode an.
				\end{compactitem}
	\end{minipage}
		
	\begin{minipage}[t]{7 cm}
		\subsection{Exceptionhandling in $C++$ \verweiscpp{15.2}}
			\begin{compactitem}
				\item Exceptions werden in Form eines Objekts am Ort ihres Auftretens ausgeworfen (explizit oder auch "automatisch").
				\item Exception Handler versuchen, diese Exception-Objekte aufzufangen.
			\end{compactitem}
			
			\subsubsection{Ausl�sen (Werfen) von Ausnahmen}
				\begin{compactitem}
					\item Ausnahmen k�nnen mit dem Schl�sselwort $throw$ explizit ausgeworfen werden.
					\item Nach einem $throw$-Befehl wird das Programm abgebrochen und beim ersten passenden umgebenden Handler fortgesetzt.
					\item Dabei werden alle lokalen Objekte wieder automatisch zerst�rt (Stack unwinding).
					\item Geworfen werden kann ein beliebiges Objekt (�blich: ein spezifisches Ausnahmeobjekt).
					\item (Ausschliesslich) innerhalb eines Exception Handlers ist auch die Form
					$throw;$ erlaubt. Dadurch wird die Exception an den	n�chsten Handler weitergereicht (Exception propagation).
				\end{compactitem}	
	\end{minipage}
	\hspace*{1cm}
	\begin{minipage}[t]{11 cm}
		\lstinputlisting[language=C++,tabsize=2]{code/exception_handling.cpp}
	\end{minipage}
	
	\begin{minipage}[t]{9 cm}
		\subsubsection{Exception-Hierarchie in $C++$}
			\includegraphics[width=1\textwidth]{pics/exception.jpg}
	\end{minipage}
	\hspace*{0.5cm}
	\begin{minipage}[t]{9 cm}
		\subsubsection{Laufzeit- vs. Logische Fehler}
			\begin{compactitem}
				\item Logische "Fehler" (logic\_error)
				\begin{compactitem}
					\item Ausnahmen im Programmablauf, die bereits zur Entwicklungszeit ihre Ursache	haben.
					\item Theoretisch k�nnten diese Ausnahmen verhindert werden.
				\end{compactitem}
				\item Laufzeit "Fehler" (runtime\_error)
				\begin{compactitem}
					\item Nicht vorhersehbare (?) Ausnahmen wie z.B. arithmetische �berl�ufe.
					\item Diese Ausnahmen treten erst zur Laufzeit auf, z.B. durch eine nicht erlaubte Benutzereingabe.
				\end{compactitem}
			\end{compactitem}
	\end{minipage}