\section{Module und Datenkapseln \verweiscpp{10}}
	\begin{minipage}[t]{7 cm}
		\subsection{Motivation \verweiscpp{10.1}}
		\begin{compactitem}
			\item Arbeitsteilung: Grosse Programme werden von mehreren Personen entwickelt. Praktikabel ist, wenn nur eine Person an einer bestimmten Datei arbeitet.
			\item Effizienz: Eine ܜbersetzungseinheit (Datei) muss bei jeder Ąnderung neu �bersetzt werden (je gr�sser die Datei desto langsamer die ܜbersetzung)
			\item Strukturierung: Ein grosses Programm in mehrere vern�nftige Teile (Baugruppen, Units) aufteilen (Divide and conquer) 
		\end{compactitem}
	\end{minipage}
	\hspace*{0.5cm}
	\begin{minipage}[t]{11 cm}
		\subsection{Nomenklatur Modul vs. Unit}
			\begin{compactitem}
				\item Ein Programmbaustein wird traditionell mit Modul (der oder das Modul) bezeichnet
				\item Der Test eines Moduls heisst folglich Modultest
				\item Das Vorgehen, welches Module generiert, heisst Modularisierung 
				\item Heute �blicher wird Modul mit Unit, der Test mit Unittest bezeichnet, das Vorgehen heisst weiterhin Modularisierung 
			\end{compactitem}
		\subsection{Ziele der Modularisierung}
			\begin{compactitem}
				\item Klare, m�glichst schlanke Schnittstellen definieren
				\item Units so bilden, das Zusammengeh�rendes in einer Unit isoliert wird (Koh�sion soll hoch sein)
				\item Schnittstellen zwischen den Units sollen klein sein (Kopplung soll klein sein)
				\item Abh�ngigkeiten unter den Units sollen eine Hierarchie bilden, zirkul�re (gegenseitige) Abh�ngigkeiten m�ssen vermieden werden
			\end{compactitem}
	\end{minipage}
	
	\begin{minipage}[t]{8 cm}	
		\subsection{Vom Modul zur Datenkapsel \verweiscpp{10.2}}
			\begin{compactitem}	
			Eigenschaften einer Unit (eines Moduls):
				\item realisiert eine in sich abgeschlossene Aufgabe
				\item kommuniziert �ber ihre Schnittstelle mit der Umgebung
				\item kann ohne Kenntnisse ihres inneren Verhaltens in ein Gesamtsystem integriert werden (include Header)
				\item ihre Korrektheit kann ohne Kenntnis ihrer Einbettung in einem Gesamtsystem nachgewiesen werden (mittels Unittest)
			Die Schnittstelle beschreibt, was das Modul zur Verf�gung stellt verbirgt dabei wie das Verhalten konkret realisiert ist (Geheimnisprinzip, Information Hiding). Der User der Unit darf keine Annahme über den inneren Aufbau machen. Der Entwickler der Unit kann deren inneren Aufbau ver�ndern solange die Schnittstelle dadurch nicht �ndert. 
			\end{compactitem}
		\subsection{Unitkonzept / Module und Datenkapseln in C++ \verweiscpp{10.3}}
			\begin{compactitem}	
			\item Interface definiert die Schnittstelle, d.h. die Deklarationen wie Funktionsprototypen, etc. (Schaufenster)
			\item Implementation: in diesem Teil sind die Unterprogramme definiert, d.h. auscodiert (Werkstatt)
			\item Das Interface wird in einer Headerdatei (*.h) beschrieben, die Implementation liegt in einer *.cpp- Datei
			\end{compactitem}
			\subsection{Die Schnittstellen-/Headerdatei \verweiscpp{10.3.1}}
			    Jede .h-Datei enthält als erste Anweisungsfolge (Umfasst die Punkte 2-7) eine Include-Guard welche Mehrfacheinfügen verhindert. Der Syntax lautet:
			    \lstinputlisting[language=C,tabsize=2]{code/includeguard.c}
			    Deklarationsreihenfolge in Headerdatei (*.h)  (Beispiel Kap 16.1)
				\begin{enumerate}
				\item Dateikommentar
				\item #include der verwendeten System-Header (iostream, etc.)
				      #include <...>
				\item #include der projektbezogenen Header (#include "...")
				\item Konstantendefinitionen
				\item typedefs und Definition von Strukturen
				\item Allenfalls extern-Deklaration von globalen Variablen
				\item Funktionsprototypen, inkl. Kommentare der Schnittstelle, bzw. Klassendeklarationen
				\end{enumerate}
			\subsection{Die Implementierungsdatei \verweiscpp{10.3.2}}
				Deklarationsreihenfolge in Implementierungsdatei (*.cpp) (Beispiel Kap 16.1)
				\begin{enumerate}
			 		\item Dateikommentar
					\item #include der verwendeten System-Header (iostream, etc.)
					      #include <...>
					\item #include der projektbezogenen Header (#include "...")
					\item Präprozessor-Direktiven
					\item typedefs und Definition von Strukturen
					\item Funktionsprototypen von lokalen, internen Funktionen
					\item Definition von Funktionen und Klassen (Kommentare aus Headerdatei nicht wiederholen!!)
				\end{enumerate}
		\end{minipage}
		
		\begin{minipage}[t]{11 cm}
			\subsection{Buildprozess / Makefile}
			    Der Buildprozess erstellt aus den einzelnen Dateien einen ausführbaren Code. Dazu werden zuerst alle *.cpp-Files compiliert.
			    Die daraus entstandenen Objektdatei müssen anschliessend gelinkt und somit zu einer auführbaren Datei zusammengesetzt. Die Eingabe in der Konsole sieht wie folt aus:
			    \lstinputlisting[language=C,tabsize=2]{code/compileUnit.c}
				\begin{compactitem}
					\item Ein Programmbaustein wird traditionell mit Modul (der oder das Modul) bezeichnet
					\item Der Test eines Moduls heisst folglich Modultest
					\item Das Vorgehen, welches Module generiert, heisst Modularisierung 
					\item Heute üblicher wird Modul mit Unit, der Test mit Unittest bezeichnet, das Vorgehen heisst weiterhin Modularisierung 
				\end{compactitem}
			\subsection{Ziele der Modularisierung}
				\begin{compactitem}
					\item Klare, möglichst schlanke Schnittstellen definieren
					\item Units so bilden, das Zusammengehörendes in einer Unit isoliert wird (Kohäsion soll hoch sein)
					\item Schnittstellen zwischen den Units sollen klein sein (Kopplung soll klein sein)
					\item Abhängigkeiten unter den Units sollen eine Hierarchie bilden, zirkuläre (gegenseitige) Abhängigkeiten müssen vermieden werden
				\end{compactitem}
		\end{minipage}
		
