\section{Polymorphismus / Mehrfachvererbung / RTTI}
   	\begin{flushleft}
   	Dieses Kapitel beschreibt die dynamischen objektorientierten Sprachmerkmale von C++. Erst durch diese wird C++ zu einer echten objektorientierten Programmiersprache.
   	\end{flushleft}
	\begin{minipage}[t]{8 cm}
		\subsection{Polymorphismus \verweiscpp{14.1}}
			\subsubsection{dynamische vs. statische Bindung}
			Werden von einer Kasse A die Klassen B und C abgeleitet, so k�nnen Objekte vom Typ $Zeiger auf A$ auch auf B- oder C-Objekte verweisen. Implementieren alle drei Klassen eine Operation foo jeweils verschieden so bewirkt die Anweisung
			\lstinputlisting[language=C++,tabsize=2]{code/foo_poly.cpp}
			in normalen Programmiersprachen den Aufruf von $A::foo()$. Dabei wird bereits zur �bersetzungszeit (so fr�h wie m�glich; Early Binding) vom Compiler die Funktion $foo$ der Klasse A eingebunden. Diese Art des Bindens wird statische Bindung (static binding) genannt, da sie unver�nderbar ist. Die Variable $anAPointer$ kann in C++ auch f�r Objekte der Klasse B oder C stehen. \linebreak
			In echten objektorientierten Programmiersprache wird der obige Aufruf nicht zur �bersetzungszeit, sondern erst zur Laufzeit gebunden (dynamische Bindung, dynamic Binding). Beim Aufruf von \lstinputlisting[language=C++,tabsize=2]{code/foo_poly.cpp} wird der Typ des Objekts untersucht. In Abh�ngigkeit davon wird die Methode $A::foo$, $B::foo$ oder $C::foo$ aufgerufen. Dieses dynamische Verhalten wird als Polymorphismus bezeichnet. Damit dynamisch (zur Laufzeit) die verschiedenen Funktionen foo aufgerufen werden k�nnen, m�ssen diese Funktionen $virtual$ sein. \linebreak
	\end{minipage}\hspace*{0.5cm}
	\begin{minipage}[t]{10.5 cm}
		\subsection{Virtuelle Elementfunktionen \verweiscpp{14.2}}
			Virtuelle Elementfunktionen sind spezielle Funktionen, die nicht zur �bersetzungs- sondern zur Laufzeit gebunden werden. Es wird erst beim Auruf der Funktion entschieden, welche tats�chlich ausgef�hrt wird $A::foo$, $B::foo$ oder $C::foo$
				\begin{compactitem}
					\item Funktionen, die dynamisch gebunden werden, muss bei der Deklaration das Schl�sselwort virtual vorangestellt werden (zwingend!).
					In der abgeleiteten Klasse soll (muss aber nicht) die Funktion auch mit virtual gekennzeichnet werden. Dies sieht wie folgt aus:
						\lstinputlisting[language=C++,tabsize=2]{code/virtual.cpp}
					\item Faustregel: Eine Funktion sollte als virtual deklariert werden, wenn sie in der abgeleiteten Klasse neu definiert (�berschrieben) wird, sonst nicht!
					\item Achtung: nicht mit Funktions�berladung (gleicher Name aber unterschiedliche Signatur) verwechseln
					\item Die neue (�berschriebene) Methode muss dieselbe Signatur wie die Methode der Basisklasse haben. Sonst wird neue Methode eingef�hrt.
					\linebreak
				\end{compactitem}
	\end{minipage}
	\begin{minipage}[t]{6.5 cm}
	Im Beispiel rechts wird die Verwendung klar:
			\begin{compactitem}
				\item Der statische Datentyp bezeichnet den Datentyp bei der Deklaration. Im Beispiel: a ist ein Array von Pointer auf Article
				\item Der dynamische Datentyp bezeichnet den effektiven Datentyp zur Laufzeit Im Beispiel: a[0] ist ein Pointer auf Book, a[1] ein Pointer auf CD, etc.
			\end{compactitem}
	\end{minipage}
	\hspace*{0.5cm}
	\begin{minipage}[t]{12 cm}
	test\linebreak
		\includegraphics[width=1\textwidth]{pics/bsp_Webshop.jpg}
	\end{minipage}
\newpage
	\subsubsection{Aufruf von virtuelle Elementfunktionen \verweiscpp{14.2.2}}
	\begin{minipage}[t]{9 cm}
		Eine dynamische Methodenaufl�sung erfolgt �ber Zeiger oder Pointer:
		\lstinputlisting[language=C++,tabsize=2]{code/dynamic_call.cpp}
	\end{minipage}\hspace*{0.5cm}
	\begin{minipage}[t]{9 cm}
		Ein Aufruf mit einem Objekt und der Punktnotation wird statisch aufgel�st:
		\lstinputlisting[language=C++,tabsize=2]{code/static_call.cpp}
		Dies kommt daher, dass ein echtes Objekt sein Typ nicht ver�ndern kann (nicht polymorph) und der Compiler somit schon zur �bersetzungszeit entscheidet welche Funktion aufgerufen wird.
	\end{minipage}
	\begin{flushleft}
		Eine statische Aufl�sung wird auch erzwungen, wenn der G�ltigkeitsbereich explizit angegeben wird:
	\end{flushleft}
		\lstinputlisting[language=C++,tabsize=2]{code/static_call2.cpp}
	\begin{flushleft}
		Wichtig ist auch: innerhalb von Konstruktoren und Destruktoren \textbf{alle} Methodenaufrufe statisch aufgel�st werden. 
	\end{flushleft} 
	\subsubsection{Aufruf von virtuelle Elementfunktionen \verweiscpp{14.2.2}}
		