\section{Klassenkonzept}
	\begin{minipage}[t]{7 cm}
		\subsection{Begriff der Klasse}
		\begin{compactitem}
			\item Eine Klasse ist eine Struktur (eine Struktur besteht nur aus Daten), die mit den 	Funktionen, welche auf diesen Daten arbeiten, erweitert wurde.
			\item Eine Klasse ist also eine Struktur, welche die Daten und die Funktionen auf diesen Daten in ein syntaktisches Konstrukt packt.
			\item Die Klasse ist die Umsetzung der Datenkapsel.
			\item Eine Klassendeklaration ist eine Typendefinition. Die Variablen einer Klasse
			werden als Objekte bezeichnet.
		\end{compactitem}
	\end{minipage}
	\hspace*{0.5cm}
	\begin{minipage}[t]{11 cm}
		\subsection{UML-Notation einer Klasse}
			\begin{tikzpicture}
				\begin{class}[text width=9.5cm]{ClassName}{0 ,0}
					\attribute {-attribute1: int = 0}
					\attribute {-attribute2: int = 0}
					\operation {+method1()}
					\operation {+method2()}
				\end{class}
			\end{tikzpicture}
			\begin{compactitem}
				\item Eine Klasse ist der Bauplan f�r Objekte.
				\item Eine Klasse besteht aus Daten (Attribute) und den Funktionen (Methoden) auf diesen Daten.
				\item Sichtbarkeit: 
				\begin{compactitem}
					\item $+$ : $public$
					\item $-$ : $private$
					\item $\#$ : $protected$
				\end{compactitem}
			\end{compactitem}
	\end{minipage}
	
	\begin{minipage}[t]{8 cm}	
		\subsection{�blicher Aufbau einer Klassensyntax \verweiscpp{11.1.1}}
			\lstinputlisting[language=C++,tabsize=2]{code/klassenschnittstelle.cpp} 
	\end{minipage}
	\hspace*{0.5cm}
	\begin{minipage}[t]{10 cm}		
			\subsubsection{Zugriffsschutz \verweiscpp{11.4}}
			\begin{compactitem}
				\item Innerhalb der Klasse hat jede Methode der Klasse auf alle Elemente Zugriff. (innerhalb der Klasse sind die Attribute "lokale globale" Datenfelder)
				\item Von ausserhalb der Klasse gibt es grunds�tzlich keinen Zugriff auf Elemente der Klasse (default, d.h. wenn nichts explizit steht). 
				\item Alles, was von aussen zugreifbar sein soll, muss mit $public:$ gekennzeichnet werden.
				\item Obwohl nicht unbedingt notwendig, werden die nach aussen nicht sichtbaren Elemente �blicherweise explizit mit $private:$ gekennzeichnet.
			\end{compactitem}
	\end{minipage}
	
	\begin{minipage}[t]{9 cm}
		\subsubsection{Operationen einer Klasse}
			Operationen eine Klasse (= Funktionen, die im Klassenrumpf definiert sind) werden als
			Elementfunktionen oder Methoden bezeichnet.	�blicherweise beginnen Elementfunktionen mit einem Kleinbuchstaben und werden in camelCase (mixedCase) notiert.	
			\begin{lstlisting}[language=C++,tabsize=2]
				isEmpty();
			\end{lstlisting}	
	\end{minipage}
	\hspace*{0.5cm}
	\begin{minipage}[t]{9 cm}	
		\subsubsection{Information Hiding}
			\begin{compactitem}
				\item Klassen exportieren generell ausschliesslich Methoden. Alle Daten sind im Innern (private-Abschnitt) verborgen, der Zugriff erfolgt �ber die so genannten Elementfunktionen.
				\item Jede Klasse besteht damit aus zwei Dateien, der Schnittstellendatei ($.h$) und	der Implementierungsdatei ($.cpp$).
			\end{compactitem}
	\end{minipage}

\newpage
		\subsubsection{Beispiel an der Klasse Rechteck}
			\begin{minipage}[t]{9cm}
				\lstinputlisting[language=C++,tabsize=2]{code/class_rectangle_header.cpp}				
				\begin{tikzpicture}
					\begin{class}[text width=8cm]{Rectangle}{0 ,0}
						\attribute{-a : double}
						\attribute{-b : double}
						\operation{+setA(in newA : double)}
						\operation{+setB(in newB : double)}
						\operation{+getA() : double}
						\operation{+getB() : double}
						\operation{+getArea() : double}
					\end{class}
				\end{tikzpicture}
			\end{minipage}
			\hspace*{0.5cm}
			\begin{minipage}[t]{9 cm}
				\lstinputlisting[language=C++,tabsize=2]{code/class_rectangle.cpp} 
			\end{minipage}