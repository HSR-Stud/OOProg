\section{Einfache Deklarationen und Basisdatentypen \verweiscpp{4}}
	\subsection{Definition und Deklaration \verweiscpp{4.1}}
	Die Begriffe Deklaration und Definition werden oft synonym verwendet. Sie bezeichnen aber
	verschiedene Dinge: Eine Deklaration f�hrt einen oder mehrere Namen in einem Programm ein. Dem Compiler werden zwar mit dem Namen Informationen �ber einen Typ oder eine Funktion bekanntgegeben, es wird aber kein Programmcode erzeugt oder Speicherplatz f�r ein Objekt angelegt. Eine Definition wiederum vereinbart konkrete Objekte im Programm, also Variablen (inklusive deren
	Speicherplatz) oder ausf�hrbaren Code. Jede Definition ist damit zugleich eine Deklaration. Ebenso sind
	sehr viele Deklarationen zugleich Definitionen.
	
	\subsection{Variablendeklaration}
	\lstinputlisting[language=C++,tabsize=2]{code/variablendeklaration.cpp}
	\subsection{Variableninitialisierung}
	Um eine Variable zu initialisieren, gibt es mehrere M�glichkeiten:
	\lstinputlisting[language=C++,tabsize=2]{code/variableninit1.cpp}
	Eine Variable muss nicht sofort mit einem Wert initialisiert werden. Es ist auch m�glich, sie zun�chst nur zu definieren und ihr sp�ter einen Wert zuzuweisen.
	\subsection{Die One Definition Rule \verweiscpp{4.2}}
		Die One Definition Rule besagt vereinfacht dargestellt, dass ein Name genau einmal in einem Programm definiert sein darf. Es gibt jedoch einen Fall, bei dem diese Regel nicht verletzt wird und man trotzdem zwei Mal denselben Namen verwenden kann:
		
		\lstinputlisting[language=C++,tabsize=2]{code/one_definition_rule.cpp}
		
		Hier liegt keine Verletzung vor. A ist in beiden Definitionen identisch und wird daher als eine einzelne Definition betrachtet. Ein Fehler l�ge dann vor, wenn die beiden Definitionen unterschiedlich w�ren.
	\subsection{Basisdatentypen \verweiscpp{4.3}}
	Basisdatentypen sind vordefinierte einfache Datentypen. Sie umfassen Wahrheitswerte (bool), Zahlen (int, short int, long int, float, double), Zeichen (char, wchar\_t) und den Typ "nichts" (void).
		\subsection{�bersicht �ber alle Standard-Datentypen \verweisc{5.2}}
				\begin{tabular}{|c|c|c|c|c|}
						\hline
							\textbf{Datentyp} & \textbf{Anzahl Bytes} & \textbf{Wertebereich (dezimal)} & Typ & Verwendung\\
						\hline
						\hline
							$char$ & 1 & $-128$ bis $+127$ & Ganzzahltyp & speichern eines Zeichens\\
						\hline
							$unsigned$ $char$ & 1 & $0$ bis $+255$ & Ganzzahltyp & speichern eines Zeichens\\
						\hline
							$signed$ $char$ & 1 & $-128$ bis $+127$ & Ganzzahltyp & speichern eines Zeichens\\
						\hline
						\hline
							$int$ & 4 (in der Regel) & $-2'147'483'648$ bis $+2'147'483'647$ & Ganzzahltyp & effizienteste Gr�sse\\
						\hline
							$unsigned$ $int$ & 4 (in der Regel) & $0$ bis $+4'294'967'295$ & Ganzzahltyp & effizienteste Gr�sse\\
						\hline
						\hline
							$short$ $int$ & 2 (in der Regel) & $-32'768$ bis $+32'767$ & Ganzzahltyp & kleine ganzzahlige Werte\\
						\hline
							$unsigned$ $short$ $int$ & 2 (in der Regel) & $0$ bis $+65'535$ & Ganzzahltyp & kleine ganzzahlige Werte\\
						\hline
						\hline
							$long$ $int$ & 4 (in der Regel) & $-2'147'483'648$ bis $+2'147'483'647$ & Ganzzahltyp & grosse ganzzahlige Werte\\
						\hline
							$unsigned$ $long$ $int$ & 4 (in der Regel) & $0$ bis $+4'294'967'295$ & Ganzzahltyp & grosse ganzzahlige Werte\\
						\hline
						\hline
							$float$ & 4 (in der Regel) & $-3.4*10^{38}$ bis $+3.4*10^{38}$ & Gleitpunkttyp & Gleitpunktzahl\\
						\hline
							$double$ & 8 (in der Regel) & $-1.7*10^{308}$ bis $+1.7*10^{308}$ & Gleitpunkttyp & h�here Genauigkeit\\
						\hline
							$long$ $double$ & 4 (in der Regel) & $-1.1*10^{4932}$ bis $+1.1*10^{4932}$ & Gleitpunkttyp & noch h�here Genauigkeit\\
						\hline
					\end{tabular}
		\subsubsection{Datentyp bool \verweiscpp{4.3.1}}
		Der Datentyp f�r Wahrheitswerte heisst in C++ bool, was eine Abk�rzung f�r boolean ist. Er kann nur zwei Zust�nde annehmen: true (wahr) oder false (falsch). Obwohl eigentlich 1 Bit ausreichen w�rde, hat bool mindestens eine Gr�sse von einem Byte (also 8 Bit), denn 1 Byte ist die kleinste adressierbare Einheit und somit die Minimalgr�sse f�r jeden Datentyp. 
		\subsubsection{Datentyp void \verweiscpp{4.3.5}}
		void ist ein spezieller Typ, der anzeigt, dass kein Wert vorhanden ist. Es ist nicht m�glich, ein Objekt
		vom Typ void anzulegen. Vielmehr findet der Datentyp Anwendung bei der Deklaration von speziellen Zeigern, von denen nicht bekannt ist, auf welchen Typ sie verweisen, oder bei Funktionen, die keinen R�ckgabewert liefern.
		\lstinputlisting[language=C++,tabsize=2]{code/void.cpp}

	\subsection{Deklaration von Konstanten \verweiscpp{4.5}}
	Eine Konstante wird deklariert, indem vor dem eigentlichen Typ das Schl�sselwort const notiert wird:
	\lstinputlisting[language=C++,tabsize=2]{code/const1.cpp}
	Wird versucht, w�hrend der Programmausf�hrung der Konstante val einen Wert zuzuweisen, so f�hrt dies zu einem �bersetzungsfehler. \\\\Wichtig: Es g�be noch eine Variante mit \#define (vorallem C). Diese Variante sollte in C++ keinesfalls verwendet werden, da nur eine textuelle Ersetzung erfolgt! 
		\subsubsection{Zeichenkonstanten}
			\lstinputlisting[language=C++,tabsize=2]{code/const2.cpp}
		\subsubsection{Integerkonstanten}
			\lstinputlisting[language=C++,tabsize=2]{code/const3.cpp}
		\subsubsection{Fliesskommakonstanten}
			\lstinputlisting[language=C++,tabsize=2]{code/const4.cpp}
	\subsection{Enumerations (Aufz�hlungstyp)}
	 	\begin{minipage}[t]{9 cm}
		 	\vspace*{-0.5cm}
		 	\lstinputlisting[language=C,tabsize=2]{code/enum1.c}
		 \end{minipage}
		 \hspace*{0.5 cm}
		 \begin{minipage}[t]{8 cm}
		 	\begin{compactitem}
		 		\item Aufz�hlungskonstanten haben einen konstanten ganzzahligen Wert.
		 		\item Die erste Konstante erh�lt den Wert $0$, die zweite $1$, etc.
		 		\item Werte k�nnen auch explizit zugewiesen werden
		 	\end{compactitem}
		\end{minipage}
		 		
		\subsubsection{Anonyme Enumerations}
			$enums$ k�nnen auch verwendet werden, um ganzzahlige symbolische Konstanten zu definieren. Der $enum$ erh�lt dann keinen Namen, er wird nur dazu verwendet, die einzelnen Konstanten festzulegen. Bessere Alternative zu $\#define$ f�r ganzzahlige Konstanten!!
		 		\lstinputlisting[language=C,tabsize=2]{code/enum2.c}
		 		
		